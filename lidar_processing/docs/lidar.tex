\documentclass[float=false, crop=false]{standalone}

\usepackage{graphicx}
\usepackage{rotating}
\usepackage{natbib}
\usepackage{longtable,lscape}
\usepackage{verbatim} 
\usepackage{algorithm}
\usepackage{algorithmic}
\usepackage[margin=2.5cm]{geometry}
\usepackage{amsmath}
\usepackage{amssymb}
\usepackage{mdwlist}
\usepackage{xcolor}
\usepackage{graphicx}
\usepackage{times,epsfig,graphicx}
\usepackage{fancyhdr}
\usepackage[hyphens]{url}
\usepackage[hidelinks]{hyperref}
\usepackage{subfigure}
\usepackage{caption}
\usepackage[T1]{fontenc}

\begin{document}

The {\bf{track\_splitter}} application splits a LOLA file into multiple tracks and saves info for each track into
a separate file.

\section{Files:}

\begin{itemize}
\item{track\_splitter.cc - The main method, parses command line arguments and splits a LOLA files in multiple
  tracks based on time stamps.}
\item{TracksLOLA.cc and TracksLOLA.h - Define, load and save track data structures. Compute track reflectance and
luminence, and transform tracks.} 
\item{TracksFeatures.cc and TracksFeatures.h - Find key points to use as ground control points.}
\item{TracksWeights.cc and TracksWeights.h - Compute track weights based on the features saliency} 
\end{itemize}

\section{How to install:}
\begin{enumerate}
	\item{\emph{Install Prerequisites}} - Install GDAL 1.9, OpenCV 2.4.9, ISIS 3.1, Eigen 3, Boost 1.5
	\item{\emph{Build }} - In the \texttt{lidar2image\_processing/tests} directory,
          run ``\texttt{cmake ..}'' followed by ``\texttt{make}'' to create the {\texttt{tests/lidar2image}}
          executable.
\end{enumerate}

\section{How to run}
The command ``\texttt{lidar\_image\_align -l lola\_tracks.csv -i image.cub}'' will        
align the LOLA tracks in the CSV file \texttt{lola\_tracks.csv} to the image file \texttt{image.cub}.
The transformation matrix and the ground control point files will be output in results directory under
filename\_transf.txt and filename\_trackIndex\_shotIndex\_gcp.txt where filename is the stem of the image
filename (filename with no path and no extension). The following options may be passed to
{\texttt{lidar\_image\_align}} at runtime.                   
\begin{itemize}
	\item{\texttt{-l, --lidarFile filename}} : a CSV file containing the LOLA shots to process or a file containing a
        list of LOLA CSV shots to process, separated by linebreaks
	\item{\texttt{-i, --inputCubFile filename}} : a cub image to align tracks to
        \item{\texttt{-r, --results directory name}} : optional parameter for the directory name where results are saved
        \item{\texttt{-s, --settings filename}} : optional parameter containing the settings filename
        \item{\texttt{-t, --test mode}} : optional parameter to run in test mode
	\item{\texttt{-h, --help}} : display a help message
\end{itemize}

lidar\_image\_align.sh is an script example to run data available in lidar2image\_processing/tests/data directory.  

\end{document}
